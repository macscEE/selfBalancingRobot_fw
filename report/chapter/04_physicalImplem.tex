\begin{chapter}{Physical Implementation}
\label{chap_04:physicalImplem}

\subsection{Mechanical structure}
Moving from the theoretical model and control design to the actual implementation, we design the physical structure of the robot
and printed with a 3D printer: all the structure is made of PLA material while the wheels are made of 
TPU to ensure better grip on the ground.

The complete structure is shown in Figure \ref{}.

\subsection{PID adjustment}

The PID controller parameters obtained in Chapter \ref{chap_02:modelling} are a good starting point for the real 
implementation, but due to the simplifications made in the model and the non-idealities of the real components,
it is necessary to adjust them experimentally.
After several tests, the final parameters used in the firmware are:
\begin{itemize}
    \item $K_P = 10$
    \item $K_I = 0.02$
    \item $K_D = 0.4$
\end{itemize}
These values ensure a good balancing performance, with a fast response to disturbances and minimal oscillations around the vertical position.

\subsection{Other adjustment}
Other parameters that needed to be adjust experimentally were for example the full-scale range of the accelerometer:
we experinced that setting it to $\pm 4g$ ensured a good compromise between sensitivity and range of measurement, while 
setting it to $\pm 2g$ resulted in a high noise measurement.

In order to use the robot without the need of a computer connected via USB, we inserted a simple buck converter to power the ESP32
from the batteries.
\end{chapter}