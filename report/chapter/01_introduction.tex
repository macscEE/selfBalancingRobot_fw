\begin{chapter}{Introduction}
\label{ch:01_introduction}

We first start by introducing the concept of a self-balancing robot, which is a type of two-wheels robot that can maintain 
its balance while standing upright: this is achieved by continuisly measure the pitch angle of the robot and consequently 
adjust the motor's speed to keep the robot balanced.
The main components of the self-balancing robot are reported in Table \ref{tab:01_componentsList}.
\begin{table}[h]
    \centering
    \begin{tabular}{|c|c|}
        \hline
        \textbf{Component} & \textbf{Description} \\
        \hline
        ESP32 & Microcontroller \\
        MPU6050 & Inertial Measurement Unit (IMU) sensor \\
        DFRobot DC 6V  & DC Motors \\
        DRV8871 & Motor Driver \\
        Molicel ... & Batteries for power supplys. \\
        \hline
    \end{tabular}
    \caption{Main components of the self-balancing robot}
\label{tab:01_componentsList}
\end{table}

\section{ESP32 Microcontroller}
The ESP32 (reported in Figure \ref{fig:01_esp32}) is a powerful microcontroller developed by Espressif Systems, widely used in IoT applications due to 
its built-in Wi-Fi and Bluetooth capabilities. It features a dual-core processor, ample memory, and various peripherals, 
making it suitable for real-time control tasks required in self-balancing robots.

We chose the ESP32 for our self-balancing robot project because of its processing power but moreover for its higher clock
speed (240 MHz) compared to other microcontrollers like Arduino Uno (16 MHz) or Arduino Mega (16 MHz). This allows faster control
loop execution, which is crucial for maintaining balance in real-time.

The microcontroller can be programmed using the Arduino IDE but we chooose for another IDE, called PlatformIO, which offers
more advanced features and better project management capabilities. This IDE can be integrated into Visual Studion Code and so 
we can take trace of all the changes with Git version control system.

\begin{figure}
    \centering
    \includegraphics[width=0.3\textwidth]{figures/01_introduction/esp32.jpg}
    \caption{ESP32 development board}
\label{fig:01_esp32}
\end{figure}
\end{chapter}

\section{MPU6050}
For the IMU sensor, we selected the MPU6050 (shown in Figure \ref{tab:01_mpu6050Ranges}) which combines a 3-axis gyroscope and a 3-axis accelerometer.
This sensor provides all the data through the I2C communication protocol. The MPU6050 range of measurement are reported 
in Table \ref{tab:01_mpu6050Ranges}.

\begin{table}[h]
    \centering
    \begin{tabular}{|c|c|}
        \hline
        \textbf{Sensor} & \textbf{Range of Measurement} \\
        \hline
        Accelerometer & ±2g, ±4g, ±8g, ±16g \\
        Gyroscope & ±250, ±500, ±1000, ±2000 °/s \\
        \hline
    \end{tabular}
    \caption{MPU6050 range of measurement}
\label{tab:01_mpu6050Ranges}
\end{table}

\begin{figure}[h]
    \centering
    \includegraphics[width=0.4\textwidth]{figures/01_introduction/mpu6050.jpg}
    \caption{MPU6050 Inertial Measurement Unit (IMU) sensor}
\end{figure}

In our application we configure the accelerometer to a range of $\pm 4g$ for reasons that will be explained in Chapter ... %metti reference a capitolo prove
and the gyroscope to a range of $ 250 \hspace{2pt} ^\circ/s$ since we need high sensitivity and we don't expect higher angular velocities.

\section{Motor Driver and DC Motors}

\subsection{DC motor}
For the DC motor we selected the DFRobot DC 6V (shown in Figure \ref{fig:01_dcMotor}), a motor that comes with a 120:1 gear ratio, providing high 
torque at low speeds, which is ideal for balancing applications. The motor is powered by a 6V power supply and
can draw a stall current of up to 1.2A.

\begin{figure}[h]
    \centering
    \includegraphics[width=0.4\textwidth]{figures/01_introduction/dcMotor.jpg}
    \caption{DFRobot DC 6V motor with 120:1 gear ratio}
\label{fig:01_dcMotor}
\end{figure}

\subsection{DRV8871}

To control the DC motors, we use the DRV8871 motor driver (shown in Figure \ref{fig:01_drv8871}), which is 
capable of handling motor supply voltages from 6.5V to 45V. It's a double full-bridge driver, allowing 
for bidirectional control of the motors.

\begin{figure}[h]
    \centering
    \includegraphics[width=0.2\textwidth]{figures/01_introduction/drv8871.jpg}
    \caption{DRV8871 motor driver}
\label{fig:01_drv8871}
\end{figure}
